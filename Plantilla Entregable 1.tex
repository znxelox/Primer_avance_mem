\documentclass[letter, 10pt]{article}
\usepackage[utf8]{inputenc}
\usepackage[spanish]{babel}
\usepackage{amsfonts}
\usepackage{amsmath}
\usepackage[dvips]{graphicx}
\usepackage{multicol,caption}
\usepackage{url}
\usepackage[top=3cm,bottom=3cm,left=2cm,right=2cm,footskip=1.5cm,headheight=1.5cm,headsep=.5cm,textheight=3cm]{geometry}
\setlength{\columnsep}{0.5cm}

\begin{document}

\begin{multicols}{2}
[
\title{Estrategias de aprendizaje para la sintonización de parámetros}
\author{Germán Marcelo Treimun Costa \\ \small{Universidad Técnica Federico Santa Maria} \\ \small{german.treimun@alumnos.usm.cl}}
\date{\none}
\maketitle
]

\begin{abstract}

\textit{Keywords}: 
\end{abstract}

\section{Introducci\'on}
El diseño de metaheurísticas es una herramienta bastante poderosa a la hora de resolver problemas de optimización duros, pero a la vez diseñar una metaheurísitca resulta ser un proceso de alta dificultad que consume mucho tiempo. Como lo ideal es obtener la mejor solución, se debe realizar una representación adecuada del problema, y  también se deben tomar en consideración los parámetros que están asociados a este, aquí mismo es donde entra lo que se conoce como sintonización de parámetros[cita]. En el campo de la optimización, la sintonización de parámetros se refiere a encontrar los mejores valores de los parámetros para así obtener la mejor solución, o una mejor a la que ya se tiene, al mismo tiempo la idea es que la sintonización no cueste más de lo que puede beneficiar, ya que es una tarea computacionalmente compleja.\\

Hoy en día existen bastantes técnicas que realizan una buena sintonización de parámetros, como lo son ParamILS \cite{Hutter2007AutomaticAC}, Revac \cite{Nannen2007RelevanceEA}, F-Race \cite{Birattari2002ARA} y SPO \cite{BartzBeielstein2005SequentialPO}, los cuales necesitan correr el algoritmo varias veces, a la vez generando bastante información, ayudando así al diseñador a tener un mejor entendimiento del algoritmo y por lo mismo mejorándolo. Uno de los grandes contras de estos algoritmos es que son bastante sofisticados y los diseñadores menos experimentados pueden verse complicados  a la hora de sintonizar parámetros. Aquí es donde entra EVOCA, un algoritmo propuesto por Riff et Al. \cite{Riff2013ANA} el cual entrega a diseñadores poco experimentados soluciones de buena calidad  de manera simple, sin necesidad de un alto conocimiento de sintonización de parámetros.[AQUI PUEDE HABER MÁS]\\

El objetivo de esta propuesta de memoria es aplicar metodologías de aprendizaje a EVOCA, para así poder entregar al diseñador de metaheurísticas información relevante, ya sea de los parámetros categóricos y/o numéricos, observar la interdependencia de estos y obtener mejores resultados a los cuales ya obtiene.
%Lo anterior no lo entiendo
Para ello en este trabajo es  importante entender como funciona EVOCA y realizar un estudio de la literatura respecto a lo que se ha hecho en el campo de la sintonización de parámetros.\\

En este trabajo se presenta en la sección 2 los objetivos preliminares. Después, en la sección 3, se presenta la definición del problema y algunas razones que justifican el querer resolverlo. En la sección 4 se presentará una definición más formal de EVOCA y cómo funciona. Posteriormente en la sección 5, se presentará un estado del arte preliminar en donde se revisarán trabajos presentes en la literatura sobre algoritmos de sintonización de parámetros. La sección 6 entrega conclusiones preliminares sobre el trabajo actual.


\section{Objetivos}

\subsection{Objetivo General}
Realizar un estudio de la literatura para evaluar y comparar distintos enfoques y algoritmos que sintonizan parámetros presentados por diversos autores con la meta de poder rescatar lo mejor de ellos y visualizar como aplicarlo junto a metodologías de aprendizaje a EVOCA.


\subsection{Objetivos Específicos}
Se tienen como objetivos específicos:
\begin{itemize}
    \item Realizar una recopilación y comparación de algoritmos de sintonización de parámetros propuestos en 	la literatura (los datos que piden de entrada, cuantas ejecuciones necesitan y la calidad de las soluciones).
    \item Analizar cómo los algoritmos monitorean los parámetros de forma individual y la interdependencia existente entre varios de estos.
    \item Entender cómo funciona EVOCA y cómo se pueden aplicar metodologías de aprendizaje al algoritmo.
\end{itemize}	

\section{Definici\'on del Problema}

La sintonización de parámetros, también conocida como meta-optimización, es el uso de un método de optimización para sintonizar otro método de optimización. Data desde los años 70 cuando Mercer y Sampson \cite{MercerSampson72} intentaron encontrar la sintonización de parámetros óptima para un algoritmo genético. Aunque se trate de un área bastante específica, la cual puede ser considerada como una capa más arriba de las metaheurísticas, realizar avances en los algoritmos de sintonización de parámetros es muy importante, pues estos afectan a la solución, ya que cada problema y las instancias de estos tienen parámetros propios, ya sean numéricos, categóricos o condicionales. Los parámetros pueden estar interrelacionados o incluso pueden variar durante la ejecución, por lo cual encontrar los mejores valores resulta ser un proceso complejo computacionalmente y que toma bastante tiempo. \\

Dado la importancia que tiene realizar una buena sintonización de parámetros, existen varios algoritmos propuestos por distintos autores, pero varios de estos a la vez requieren de un conocimiento en profundidad de la sintonización, y es aquí donde el diseñador de metaheurísticas puede verse un poco trabado. EVOCA busca de manera simple ayudar a los diseñadores que cuentan con poca experiencia en sintonización de parámetros, entregando buenas soluciones, que están casi al mismo nivel que algoritmos más sofisticados y de hecho en algunos casos son mejores que otros. EVOCA como muchos otros sintonizadores debe ejecutarse muchas veces para poder sintonizar. El objetivo final de esta propuesta de memoria es sacar provecho de todas las veces que se ejecutó el algoritmo para guiar el proceso de sintonización, es decir, obtener información del proceso, ya sea si un parámetro por más que pueda variar no afecta mucho a la solución, o si al cambiar el valor de otro parámetro el parámetro anterior si afecta más a la solución, es decir la interdependencia de estos mismos. El cómo se pueda realizar lo mencionado es a través de técnicas de aprendizaje, para así en cada iteración (ejecución) del algoritmo se pueda ir obteniendo la información relevante mencionada anteriormente.

\section{Marco Teórico}
Para poder realizar un correcto análisis de la literatura es necesario contextualizar un poco sobre la sintonización de parámetros, los tipos que existen y es necesario desarrollar una definición más formal de EVOCA para entender como funciona.\\

Existen distintos tipos de sintonización de parámetros, existe la sintonización hecha a mano, en donde el diseñador observa el rendimiento del algoritmo y va modificando los valores de los parámetros para obtener un resultado más óptimo. La sintonización por analogía por otro lado consiste en la idea de seguir pautas creadas por autores reconocidos. Otro tipo de sintonización es la basada en el diseño experimental, en la cual como su nombre lo indican, utiliza el diseño experimental para asignar los valores de los parámetros, ejemplo de esto son los métodos de carreras. Existen también los métodos de sintonización basados en búsqueda, como lo son ParamILS y Revac. Finalmente no encontramos con los métodos híbridos, los cuales combinan métodos anteriormente nombrados, como lo hace el conocido Calibra [CITA].



\section{Estado del Arte}
A continuación se presentan algunos trabajos presentes en la literatura junto con los enfoques seleccionados por ellos, y qué resultados se obtuvieron.\\

En el año 2014 Montero et al. \cite{Montero2014ABG} presentaron un documento como guía para principiantes en los métodos de sintonización de parámetros, haciendo mención en los tipos de sintonización y además realizando un buen resumen de los métodos F-Race \cite{Birattari2002ARA}, Revac \cite{Nannen2007RelevanceEA}, ParamILS \cite{Hutter2007AutomaticAC} y SPO \cite{BartzBeielstein2005SequentialPO}, por lo cual en este documento se utilizará el trabajo hecho en \cite{Montero2014ABG} para describir y comparar los métodos recientemente mencionados:
\begin{itemize}
    \item \textbf{F-Race:} 
    Este algoritmo presentado por Bitrattari et al.\cite{Birattari2002ARA} en el año 2002 consiste en un proceso iterativo, que en cada iteración evalúa un conjunto de configuraciones candidatas, en una nueva instancia del problema o una nueva semilla y luego se elimina del conjunto la configuración que muestre el peor rendimiento estadístico. F-Race utiliza \textit{Friedman two-way analysis of variance by ranks} para comparar el rendimiento entre el conjunto de configuraciones de parámetros candidatas. Este método define tres parámetros: la cantidad de ejecuciones sin eliminación $r$, el nivel de confianza de las pruebas de hipótesis $\alpha$ y el número máximo de ejecuciones del algoritmo sintonizado $budget$. F-Race comienza con un conjunto $C$ de configuración de parámetros. El algoritmo requiere la definición de un conjunto discreto de valores para cada parámetro a sintonizar. Luego el algoritmo realiza $r$ ejecuciones de cada parámetro en $C$ para recolectar información suficiente antes de realizar una eliminación, la información se almacena en un arreglo de costos de cada parámetro. Posteriormente F-Race selecciona de manera aleatoria una instancia para continuar con las comparaciones de rendimiento entre las configuraciones de parámetros candidatas en $C$. F-Race evalúa el algoritmo sintonizado en la instancia y agrega esta información al arreglo de costos de cada parámetro. El algoritmo realiza la prueba de Friedman y si se rechaza la hipótesis nula, se puede interpretar que al menos una configuración candidata tiende a mostrar un mejor rendimiento que al menos otra configuración.Se hacen comparaciones por pares entre las configuraciones de los parámetros en donde los que tienen un rendimiento estadísticamente inferior se eliminan del conjunto configuraciones candidatas. F-Race termina cuando queda solo una configuración candidata o cuando se alcanza el máximo de ejecuciones predefinidas. 
    \item \textbf{Revac:} Este método es definido como una estimación del algoritmo de distribución \cite{Pelikan2002ASO}. Revac funciona con una población la cual esta dada por una matriz con $M$ filas, en la cual cada fila es una configuración de parámetros y contiene $k$ elementos, correspondientes a los $k$ parámetros a sintonizar. Para cada parámetro Revac inicia la búsqueda con una distribución uniforme de los valores dentro de un rango dado. En cada paso, mediante el uso de operadores de transformación especialmente diseñados, se reduce el rango de valores mencionado para cada parámetro. Por otro lado si un parámetro es más relevante que otro se establece en base a la entropía de los parámetros, de manera que a más alta entropía, menos relevancia tiene y si tiene una entropía baja, el parámetro es más relevante. El algoritmo se inicia con una población aleatoria de $M$ configuraciones de parámetros, se evalúa cada configuración de parámetros considerando solo una semilla aleatoriamente, en cada iteración solo se crea una nueva configuración de parámetros y la configuración hija, se crea mediante un cruzamiento multi-parental y transformación de mutación. El cruzamiento multi-parental considera las mejores $N<M$ configuraciones de parámetros. El operador de mutación calcula para cada parámetro un intervalo de mutación, luego se selecciona un valor aleatorio del intervalo de mutación y se asigna la configuración de un hijo.
    \item \textbf{ParamILS:}
    \textit{Parameter Iterated Local Search method} como su nombre lo indica, funciona como un algoritmo de búsqueda local iterativo. Este algoritmo propuesto en 2007 por Hutter et al. \cite{Hutter2007AutomaticAC} comienza con una configuración de parámetros por defecto, usualmente basado en la experiencia de usuario, luego el algoritmo ejecuta $R$ intentos de búsqueda para una nueva configuración de parámetros de mejor calidad que la por defecto. En cada iteración el algoritmo ejecuta $s$ perturbaciones aleatorias en la configuración a mano, luego se ejecuta. \textit{IterativeFirstImprovement(c,N)} (método de búsqueda local que busca una mejor configuración aleatoria en el vecindario de configuración de parámetros) utilizando la mejor configuración obtenida y se compara con la mejor configuración de parámetros encontrada hasta ahora. El método cuenta con una probabilidad de reinicio, para que se pueda explorar y no solo realizar explotación.
    \item \textbf{SPO:}
    Presentado por T.Bartz-Bielstein et al. \cite{BartzBeielstein2005SequentialPO} en el año 2005, SPO consiste en un método que realiza iterativamente un análisis experimental de un conjunto de puntos de diseño (configuraciones de parámetros), luego estima el rendimiento de algoritmo para sintonizar mediante un modelo de proceso estocástico, y finalmente determina los nuevos puntos de diseño. el algoritmo parte con la construcción de un conjunto de puntos de diseño mediante un diseño de \textit{Latin Hipercube Sampling} (LHS) en el rango de valores definido para los parámetros. Los $n$ puntos de diseño se eligen aleatoriamente de $n$ intervalos, los cuales a su vez se obtienen dividiendo $n$ veces el rango de cada parámetro. 
\end{itemize}

Los autores realizaron experimentos con los 4 métodos descritos y compararon los resultados, obteniendo ventajas y desventajas de cada método respecto a la calidad de los resultados, el esfuerzo que requieren los métodos, la información entregada, la usabilidad, que tan amigables son con el usuario y el comportamiento bajo distintos escenarios. Se utilizo un algoritmo genético estándar (SGA) para optimizar ocho funciones (Sphere, Rastrigin, Griewank, entre otras), con elitismo siempre presente, el criterio de \textit{fitness} para SGA es la cantidad de evaluaciones para encontrar el óptimo, el cual esta fijado para cada una de las funciones. Cada ejecución del algoritmo genético estándar realiza un máximo de $10^{5}$ evaluaciones de función. Para los experimentos los autores de este documento plantearon 3 diferentes escenarios, cada uno formado por instancias de entrenamiento y de prueba. Finalmente en las conclusiones de \cite{Montero2014ABG} se realiza un resumen de los contenidos vistos en el documento, aquí los autores mencionan que dependiendo el enfoque se debe elegir el algoritmo adecuado. Sobre la naturaleza de los parámetros, las conclusiones dicen que tanto F-Race, ParamILS y SPO pueden buscar parámetros numéricos y categóricos, mientras que Revac no realiza búsqueda parámetros categóricos, por otro lado los requerimientos de entrada de Revac al igual que SPO solo necesitan la definición del rango de valores, mientras que en este aspecto ParamILS necesita una definición discreta de los valores, al igual que F-Race, donde para este ultimo resulta ser una tarea bastante ardua. Revac, ParamILS y SPO, con métodos de naturaleza estocástica, por lo que al ejecutarlos varias veces, tienden a obtener resultados similares, diferente es el caso de F-Race el cual depende de la semilla. Otro punto a considerar es la interacción entre parámetros, aquí F-Race, ParamIlS y SPO trabajan con la noción de configuración, más que la noción de valores de parámetros, es por ello que colocan más atención en la interacción de parámetros. Sobre la salida esperada en el documento se definen 3 tipos de usuarios, para los cuales la salida esperada es distinta. El primer tipo de usuario es al cual le interesa comparar nuevas metaheurísticas respecto a otras, aquí ParamILS y SPO pueden ser buenas opciones. El segundo tipo de usuario es el que quiero explorar y analizar las capacidades de las metaheurísticas, el conjunto final de configuraciones de F-Race es una fuente importante de imformación, por lo cual lo hace una buena opción. Para el tercer tipo de usuario, que busca encontrar una buena configuración de parámetros para el problema, los autores recomiendan utilizar ParamILS, el cual realiza una búsqueda intensa y fuerte en el espacio de parámetros y encuentra de manera rápida buenas configuraciones. Sobre la velocidad de los métodos se puede observar en la tabla [INSERTAR TABLA] los valores obtenidos en el documento para cada uno de los métodos en los 3 escenarios descritos. Finalmente como criterio de termino, el único método que es capaz de terminar por si solo detectando si hay una configuración restante es F-Race, pero ParamILS y SPO pueden ser detenidos en cualquier momento entregando buenas soluciones de igual manera. \\



Otro método de sintonización es el propuesto por Johann Dréo et al. \cite{Dro2009UsingPF} el año 2009, quienes dicen que la sintonización de parámetros debe considerarse como un problema multi-objetivo a la hora de resolver metaheurísticas estocásticas. Los autores utilizan lo que ellos llaman `Frentes de rendimiento' el cual consiste en una colección de conjuntos de parámetros no dominados que satisfacen como objetivo, la velocidad (tiempo de ejecución) o el óptimo. En el documento se desarrolla una idea ya conocida, la cual dice que en los problemas de metaheurísticas estocásticas, los objetivos de velocidad y encontrar el óptimo son contradictorios, por lo tanto para una instancia del problema y de metaheurística dada, existe una colección de conjuntos de parámetros no dominados que satisfacen por un lado la velocidad, cómo la precisión del óptimo por otro lado. El método multi-objetivo de ajuste de parámetros propuesto por los autores permite agregar los otros parámetros en uno solo, el cual es la posición en el `frente de parámetro', desde un comportamiento de producción, el cual es más veloz y menos óptimo, a un comportamiento de concepción que al contrario del anterior es lo más óptimo y menos veloz. Para los experimentos se utilizaron 2 problemas: \textit{Metaheuristics, continuous problem} y \textit{Hybrid Evolutionary Algorithm, hard combinatorial problem}. Para el primero se utilizaron los parámetros de distintas metaheurísticas, como la temperatura de un \textit{Simulated Anealing} (SA), el tamaño de la población de un algoritmo evolutivo para problemas continuos, entre otros. Se utilizo el algoritmo NSGA-II \cite{Deb2002AFA} con criterio de termino de $10.000$ evaluaciones. Para el segundo se utilizaron parámetros relacionados a operadores de mutación del metodo \textit{Divide and Evolve} \cite{Schoenauer2006DivideandEvolveAN}, usado para resolver problemas de planeo temporal. Para el primer experimento se obtuvo la identificación de correlación entre parámetros y los objetivos, además los autores notaron que uno puede preferir un algoritmo diferente, dependiendo del uso especifico, en el caso de ellos con su experimento, SA resultaba ser la mejor opción si se quería optar por un algoritmo veloz. Para el segundo experimento la relación entre parámetros y objetivos es débil. Este ultimo experimento mostró también que el tiempo necesario para un conjunto de parámetros en un método de optimización compleja es bastante alto. Tras los resultados los autores concluyeron que su método es bastante adecuado para establecer parámetros de metaheurísticas, y elegir un una técnica a utilizar dependiendo del objetivo que se tenga, pero si se trabaja con muchos parámetros el costo computacional del `frente de rendimiento' es bastante alto.\\


En el año 2010, S.K Smith et al.\cite{Smit2010AnMM} propusieron un algoritmo de sintonización de parámetros multi-función, llamado M-FETA el cual esta basado en un algoritmo evolutivo multi-objetivo. Este algoritmo es capaz de aproximar la `frontera de pareto' de parámetros de forma efectiva. El interés de los autores son los valores de parámetros `robustos' que hacen que el AE (Algoritmo Evolutivo) que los utiliza sea más `generalista' que `especialista'. Los autores utilizaron vectores para las soluciones candidatas, y la calidad de estos depende del rendimiento de un AE en una colección de funciones $F=\{f_{1},...,f_{M}\}$. Al ser los AE estocásticos, se puede observar ruido en el rendimiento, y para mejorar la estimación de los parámetros, debería repetirse la medición con AE, pero esto es bastante costoso, por lo que los autores decidieron utilizar el concepto de vecindario. Para un vector $\bar{x}$ se tiene un vecindario $N_{\bar{x}}$ con $k$ individuos de menor distancia euclidiana a $\bar{x}$, esto es para validar el vector $\bar{x}$ mediante su vecindario. La idea es que si se tienen 2 vectores $\bar{x}$ e $\bar{y}$, se dice que el primero domina al otro si y solo si:
\begin{itemize}
    \item $\exists f \in F$ tal que el rendimiento del AE en $f$ basado en la data perteneciente a $N_{\bar{x}}$ es significativamente mejor que el rendimiento del AE en la data perteneciente a $N_{\bar{y}}$.
    \item $\forall e \in F (g \neq f)$ el rendimiento del AE en g para vectores en $N_{\bar{y}}$ no es significativamente mejor que el rendimiento para vectores en $N_{\bar{x}}$.
\end{itemize}
Basado en el dominio, los autores pudieron hacer un \textit{rank} de los vectores de parámetros y realizar comparaciones entre estos. Se tiene que un vector $N_{\bar{x}}$ es mejor que un vector $N_{\bar{y}}$, cuando el \textit{rank} de $N_{\bar{x}}$ es menor al de $N_{\bar{y}}$, o cuando ambos tienen el mismo \textit{rank}, pero $N_{\bar{x}}$ esta más aislado que $N_{\bar{y}}$ (donde $\bar{x}$ es mas aislado que $\bar{y}$ si, esta más lejos de sus vecinos que $\bar{y}$). El sistema propuesto por los autores tiene dos propiedades importantes desde la perspectiva de muestrear nuevos vectores para ser probados. La primera propiedad trata de un sesgo inherente para preferir vectores aislados como padres en M-FETA. Este sesgo viene del hecho de que los vecinos de un cierto vector están bastante lejos, lo cual resulta en una desviación estándar grande y esto a su vez hace que el \textit{rank} disminuya, en consecuencia, la probabilidad de ser elegido para reproducción aumenta. La segunda propiedad es que el vector hijo probablemente esté cerca del padre, por lo cual disminuye la distancia en el vecindario y en consecuencia agudiza la estimación de su utilidad. Posteriormente todos los vectores de parámetros que no son dominados, conforman el conjunto de parámetros de \textit{pareto}, es decir, no hay otro vector que obtenga un rendimiento significativamente mejor en una de las funciones de prueba. Cada vector en el conjunto de \textit{pareto} puede ser visto como `robusto', pero todos ellos representan distintas compensaciones respecto al rendimiento. La opción a elegir en esta frontera está sujeta a las preferencias de cada usuario. Ejemplos de esto, es cuando se quiere que todos los objetivos tengan el mismo peso y se elige el vector que tiene mejor rendimiento, o cuando se elige el vector el que tiene mejor rendimiento en un problema, mientras mantiene un rendimiento mínimo dado en otro problema. La configuración experimental consiste en 3 capas de arquitectura, en la capa de aplicación se utilizaron dos funciones de prueba de 10 dimensiones, Sphere[CITA] y Rastrigin[CITA]. Para Rastrigin se permitieron 8000 evaluaciones y para Sphere 4000. En la capa del algoritmo se utilizó un algoritmo genético simple con \textit{N-point Crossover}, mutación de \textit{bitflip, K-tournamen} para la selección de los padres y selección de supervivencia selectiva. En la capa del diseño, M-FETA es usado para la sintonización de parámetros. En los resultados el acercamiento multi-objetivo permitió identificar vectores específicos y varios pseudo generalistas o globales, en lugar de un solo vector. Además este enfoque da la idea de las interacciones entre los rendimientos en las funciones de ciertas pruebas, y puede dar indicaciones sobre la solidez del algoritmo sintonizador, lo cual no podría haberse obtenido utilizando el enfoque de objetivo único.\\


En febrero del año 2011 los autores  Lopéz et al. publicaron el paper \textit{The irace Package: Iterated racing for Automatic Algorithm Configuration}\cite{LpezIbez2011TheIP} el cual habla de una extensión del método \textit{F-race}. Su propósito principal es el de configurar automáticamente algoritmos de optimización para encontrar el conjunto más adecuado de parámetros para problemas de optimización. \textit{Irace} implementa una ''carrera'' de forma iterativa, la cual incluye el método \textit{I/F-Race} como un caso especial, y consiste en 3 pasos: 
\begin{enumerate}
    \item Tomar una muestra con nuevas configuraciones según una distribución particular.
    \item Seleccionar las mejores configuraciones de las muestras anteriormente tomadas por medio de carreras.
    \item Actualizar la distribución para sesgar el muestreo hacia las mejores configuraciones.
\end{enumerate}

Estos pasos se repiten hasta que se cumple un criterio de término. En \textit{irace}, cada parámetro tiene una distribución independiente, (distribución normal para parámetros numéricos o una distribución discreta para parámetros categóricos). La actualización de las distribuciones consiste en la modificación de estas (la media y la desviación estándar en el caso de la distribución normal, o los valores de probabilidad discretos de las distribuciones discretas).

Como \textit{input} el algoritmo necesita un conjunto de instancias de muestreo, un espacio de parámetros y una función de costos. Además, se necesita un número de iteraciones (carreras) que ejecutará el método. Cada carrera comienza a partir de un conjunto de configuraciones candidatas. Luego, este número disminuye con el número de iteraciones, lo que significa que se realizarán más evaluaciones por configuración en iteraciones posteriores.

En la primera iteración, las primeras configuraciones candidatas se generan al muestrear uniformemente el espacio de parámetros entregados como \textit{input}. Cuando se inicia una carrera, cada configuración se evalúa en la primera instancia mediante la función de costos. Luego, se realiza una prueba estadística de los resultados. Si hay suficiente evidencia estadística para identificar que algunas configuraciones candidatas funcionan peor que otra configuración, entonces se eliminan de la carrera, mientras que las demás (candidatos supervivientes) se ejecutan en la siguiente instancia. En cada iteración, se van modificando los valores de la media y la varianza, con el objetivo de que los valores muestreados se acerquen cada vez más al valor de la configuración principal, enfocando la búsqueda alrededor de los mejores parámetros encontrados a medida que aumenta el contador de iteraciones.

Para los experimentos de \textit{irace} se usaron \textit{soft-restart} para así evitar estancarse en óptimos locales. En la propuesta original de \textit{I/F-Race}, la desviación estándar (en el caso de los parámetros numéricos) o la probabilidad discreta (en el caso de los categóricos) disminuye en cada iteración. Sin embargo, si el ajuste converge a unas pocas configuraciones muy similares en pocas iteraciones, se pierde la diversidad y las configuraciones candidatas recién generadas no serán muy diferentes de las ya probadas, sin explorar nuevas alternativas.

Para esto, el mecanismo \textit{soft-restarts} verifica el estancamiento después de generar nuevos conjuntos de configuraciones candidatas. Consideramos que existe una convergencia prematura cuando la "distancia'' entre dos configuraciones candidatas es cero.

Como experimento, se buscó optimizar los parámetros que recibía el algoritmo \textit{Simulated Annealing}, mostrando claramente mejores resultados con los parámetros optimizados que los configuración de parámetros colocados por defecto en el software R.





\section{Conclusiones}

 
\section{Bibliograf\'ia}

\bibliographystyle{plain}
\bibliography{Referencias}

\end{multicols}

\end{document} 